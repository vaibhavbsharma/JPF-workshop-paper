\documentclass{acm_sen_article}

\usepackage{booktabs} % For formal tables
\usepackage{listings}
\usepackage{subfig}
\usepackage{url}
\usepackage{color}
\usepackage{authblk}
\renewcommand{\baselinestretch}{0.94}

% Copyright
%\setcopyright{none}
% \setcopyright{acmcopyright}
%\setcopyright{acmlicensed}
% \setcopyright{rightsretained}
%\setcopyright{usgov}
%\setcopyright{usgovmixed}
%\setcopyright{cagov}
%\setcopyright{cagovmixed}


% DOI
% \acmDOI{10.475/123_4}

% ISBN
% \acmISBN{123-4567-24-567/08/06}

%Conference
% \acmConference[]{JPF Workshop}{Nov 2017}{Urbana-Champaign, IL USA}
% \acmYear{2017}
% \copyrightyear{2017}

%\acmPrice{15.00}


\begin{document}

\lstset{language=Java}

\lstdefinestyle{nonumbers}
{numbers=none}

\newcommand{\mike}[1]{\textcolor{red}{#1}}

\definecolor{mygreen}{rgb}{0,0.4,0}
\definecolor{mygray}{rgb}{0.5,0.5,0.5}
\definecolor{mymauve}{rgb}{0.58,0,0.82}
\lstset{ %
  backgroundcolor=\color{white},   % choose the background color; you
%must add \usepackage{color} or \usepackage{xcolor}
  basicstyle=\ttfamily\small,        % the size of the fonts that are used
%for the code
  basewidth = {.5em, 0.5em},
  breakatwhitespace=false,         % sets if automatic breaks should
%only happen at whitespace
  breaklines=true,                 % sets automatic line breaking
  captionpos=b,                    % sets the caption-position to bottom
  commentstyle=\color{mygreen},    % comment style
  deletekeywords={...},            % if you want to delete keywords from
%the given language
  escapeinside={\%*}{*)},          % if you want to add LaTeX within
%your code
  extendedchars=true,              % lets you use non-ASCII characters;
%for 8-bits encodings only, does not work with UTF-8
  frame=single,	                   % adds a frame around the code
  keepspaces=true,                 % keeps spaces in text, useful for
%keeping indentation of code (possibly needs columns=flexible)
  keywordstyle=\color{blue},       % keyword style
  language=C,                 % the language of the code
  otherkeywords={*,...},           % if you want to add more keywords to
%the set
  numbers=left,                    % where to put the line-numbers;
%possible values are (none, left, right)
  numbersep=5pt,                   % how far the line-numbers are from
%the code
  numberstyle=\tiny\color{black}, % the style that is used for the
%line-numbers
  rulecolor=\color{black},         % if not set, the frame-color may be
%changed on line-breaks within not-black text (e.g. comments (green
%here))
  showspaces=false,                % show spaces everywhere adding
%particular underscores; it overrides 'showstringspaces'
%  showstringspaces=false,          % underline spaces within strings
%only
  showtabs=false,                  % show tabs within strings adding
%particular underscores
  stepnumber=1,                    % the step between two line-numbers.
%If it's 1, each line will be numbered
  stringstyle=\color{mymauve},     % string literal style
  tabsize=2,	                   % sets default tabsize to 2 spaces
%  title=\lstname                   % show the filename of files included
%with \lstinputlisting; also try caption instead of title
  literate={->}{$\rightarrow$}{2}
           {α}{$\alpha$}{1}
           {δ}{$\delta$}{1}
}

\title{Veritesting Challenges in Symbolic Execution of Java}
\author[1]{Vaibhav Sharma}
\author[1]{Michael W. Whalen}
\author[1]{Stephen McCamant}
\author[2]{Willem Visser}
\affil[1]{University of Minnesota, Minneapolis, MN, United States of
America }
\affil[2]{University of Stellenbosch, Stellenbosch, South Africa}
\affil[1]{\textit {\{vaibhav,mwwhalen\}@umn.edu},
\textit{mccamant@cs.umn.edu}}
\affil[2]{\textit{wvisser@cs.sun.ac.za}}
% \author{Vaibhav Sharma}
% \affiliation{%
%   \institution{University of Minnesota}
%   \city{Minneapolis}
%   \state{MN}
%   \postcode{55455}
% }
% \email{vaibhav@umn.edu}
% 
% \author{Michael W. Whalen}
% \affiliation{%
%   \institution{University of Minnesota}
%   \city{Minneapolis}
%   \state{MN}
%   \postcode{55455}
% }
% \email{whalen@umn.edu}
% 
% \author{Stephen McCamant}
% \affiliation{%
%   \institution{University of Minnesota}
%   %\streetaddress{P.O. Box 1212}
%   \city{Minneapolis}
%   \state{MN}
%   \postcode{55455}
% }
% \email{mccamant@cs.umn.edu}
% 
% \author{Willem Visser}
% \affiliation{%
%   \institution{University of Stellenbosch}
%   \city{Stellenbosch}
%   \country{South Africa}
% }
% \email{wvisser@cs.sun.ac.za}
% \renewcommand{\shortauthors}{V. Sharma et al.}


\maketitle
\begin{abstract}
Scaling symbolic execution to industrial-sized programs is an important open research problem.
%
Veritesting is a promising technique that improves scalability by combining the advantages of static symbolic execution with those of dynamic symbolic execution.  The goal of veritesting is to reduce the number of paths to explore in symbolic execution by creating formulas describing regions of code using disjunctive formulas.
%
In previous work, veritesting was applied to binary-level symbolic execution.

Integrating veritesting with Java bytecode presents unique challenges,
notably, incorporating non-local control jumps caused by runtime polymorphism, exceptions, native calls, and dynamic class loading.
%
If these language features are not accounted for, we hypothesize that the static code regions described by veritesting are often small and may not lead to substantial reduction in paths.  We examine this hypothesis by running a Soot-based static analysis on six large open-source projects used in the Defects4J collection.
%
We find that while veritesting can be applied in thousands of regions, allowing static symbolic execution involving non-local control jumps amplifies the performance improvement obtained from veritesting.
%
We hope to use these insights to support efficient veritesting in Symbolic PathFinder in the near future.  Toward this end, we briefly address some engineering challenges to add veritesting into SPF.
\end{abstract}

%
% The code below should be generated by the tool at
% http://dl.acm.org/ccs.cfm
% Please copy and paste the code instead of the example below.
%
% \begin{CCSXML}
% <ccs2012>
%  <concept>
%   <concept_id>10010520.10010553.10010562</concept_id>
%   <concept_desc>Computer systems organization~Embedded systems</concept_desc>
%   <concept_significance>500</concept_significance>
%  </concept>
%  <concept>
%   <concept_id>10010520.10010575.10010755</concept_id>
%   <concept_desc>Computer systems organization~Redundancy</concept_desc>
%   <concept_significance>300</concept_significance>
%  </concept>
%  <concept>
%   <concept_id>10010520.10010553.10010554</concept_id>
%   <concept_desc>Computer systems organization~Robotics</concept_desc>
%   <concept_significance>100</concept_significance>
%  </concept>
%  <concept>
%   <concept_id>10003033.10003083.10003095</concept_id>
%   <concept_desc>Networks~Network reliability</concept_desc>
%   <concept_significance>100</concept_significance>
%  </concept>
% </ccs2012>
% \end{CCSXML}
%
% \ccsdesc[500]{Computer systems organization~Embedded systems}
% \ccsdesc[300]{Computer systems organization~Redundancy}
% \ccsdesc{Computer systems organization~Robotics}
% \ccsdesc[100]{Networks~Network reliability}

\keywords{multi-path symbolic execution; veritesting; Symbolic
PathFinder; static analysis}


\section{Introduction}
Introduce veritesting, present our simple example and its performance improvement, segue into how veritesting at Java bytecode level is different from binary level
\input{challenges}
\section{Implementing Veritesting for Java}
Implementing veritesting with a symbolic execution engine for Java
source code or Java bytecode requires making a few design choices.
%
Any such implementation would be required to use a static analysis tool
for creating predicates which represented multi-path regions.
%
Another design choice is whether the predicate construction should be
performed in an online or in offline manner.
%
We explore these two questions in the following subsections.
%
\subsection{Soot-based analysis for veritesting}
%
Veritesting requires static construction of
predicates of a multi-path region which represent changes to the path expression of the dynamic
symbolic executor.
%
It also requires construction of a Control Flow Graph~(CFG) of method bodies
from Java bytecode and finding exit points of the region, which in turn
requires creation of a CFG of the region.
%
Implementing veritesting is made simpler by using a Static Single
Assignment~(SSA)~\cite{ssa} representation of the multi-path region.
%
Using an SSA form allows us to use the $\phi$-expressions created by the
SSA form and translate them into points at the end of the veritesting
region where updates to system state along different paths in the region
can be merged.
%
Soot~\cite{soot} is a static analysis framework for Java programs that
has both these features, with
ExceptionalUnitGraph~\cite{exceptionalunitgraph} and the Shimple
IR~\cite{shimple}.
%
For simple regions with only one exit point, like the one presented in Listing~\ref{lst:v_ex}, we
were able to use Soot to automate static construction of the predicate representing
an update to the expression. 
%
For doing this, we used nodes with more than one successor as the
starting point, found the immediate post-dominator of the starting
point, and traversed the CFG on all sides of such branches.
%
During such a traversal, we constructed predicates representing the
multi-path region, similar to the ones presented in
Listing~\ref{lst:v_ex_smt2}.
%
As explained in Section~\ref{sec:exit_points}, including virtual
function invocations in the construction of our predicates amplifies the
benefits of veritesting even further.
%
We plan to automate this inclusion in the future using Soot.
%
Providing SPF with updates to be made to its symbolic store also
requires Soot to maintain stack location information for variables.
%
We plan to automate SPF\rq s symbolic store updates using Soot in the
future.
%
\subsection{Integrating Veritesting with Symbolic PathFinder}
%
Integration of veritesting requires changing Symbolic PathFinder so that it can 
use a Soot-based analysis.
%
We present the sequence of actions SPF must take to implement veritesting in 
Figure~\ref{fig:spf_veritesting}.
%
The source code in Listing~\ref{lst:v_ex} was modified to add a {\tt else return;} 
statement on line 7 to create the Java bytecode shown in Figure~\ref{fig:spf_veritesting}. 
%
This integration assumes our prior Soot-based analyis provides SPF 
with a table that maps instruction offsets~(representing the start of a 
veritesting region) to a set containing (1) the multi-path region predicate to be 
added to the path expression as a conjunction, (2) symbolic store
updates, (3) exit points, (4) the expression to branch on to one of the exit points.
%
Using SPF\rq s listener mechanism, we add a listener~(named SMPListener in Figure~\ref{fig:spf_veritesting}) which listens for instructions that are starting points for a Symbolic Multi-Path~(SMP) region.
%
On finding such an instruction, \textit{SMPListener} 
(1) updates the path expression, which may involve using the symbolic stack and/or heap, 
(2) updates the symbolic stack and heap, 
(3) creates a branch~(using SPF's \textit{PCChoiceGenerator}) to 
jump to one of the exit points~(which are instructions at offsets 40 and
41 in Figure~\ref{fig:spf_veritesting})
%
SPF will then continue plain symbolic execution.
%
Thus, veritesting causes SPF to explore fewer branches.
%
For example, SPF only explores a two-way branch in Figure~\ref{fig:spf_veritesting}.
%
\begin{figure}[]
\caption{Veritesting with Symbolic PathFinder}
\label{fig:spf_veritesting}
\includegraphics[width=\columnwidth]{figures/spf_veritesting}
\end{figure}


\section{Related Work}
Talk about veritesting in Mayhem, Angr.
Talk about other symbolic execution performance improvements.
%mentioned in Christopher Kruegel's ISSTA keynote talk as Static Analysis support
LESE - loop extended sym. exec. 
  - intelligent loop unrolling
Code summarization (Dodo)
  - automatically (and statically) summarize effect of loops / functions
VSA - value set analysis
  - resolve ranges (and conditionals) without solving constraints

Talk about TamiFlex, and how using the same technique as TamiFlex is one way to solve veritesting challenges in Java bytecode.
\section{Conclusion}
%Talk about veritesting is important, but implementing it at the Java bytecode requires solving different research and engineering challenges.
Veritesting provides a way to address the scalability challenges faced by dynamic symbolic execution.
%
Nevertheless, integrating veritesting with symbolic execution at the Java bytecode level presents unique challenges.
%
Our Soot-based analysis of six large open-source Java projects confirms that, while a number of opportunities to apply veritesting arise organically in Java bytecode, the benefit of veritesting can be amplified by adding ``just in time'' support of virtual function invocations and techniques for reducing the number of exit points introduced by exceptions and early return statements.  In addition, careful integration with Symbolic PathFinder is necessary to achieve good performance.
%
%Implementing veritesting with the Symbolic PathFinder motivates robust implementation of expression reuse and introduction of complex expressions.
%
%Using veritesting to its full extent during symbolic execution of Java bytecode requires us to address such research and engineering challenges.
%
%But, veritesting has clear potential to provide significant speed-up to symbolic execution tools such as Symbolic PathFinder.

\section{Acknowledgements}
We wish to thank the Google Summer of Code program for supporting this research. 
%
This material is based on work supported by the National Science Foundation under Grant Number 1563920.


\bibliographystyle{ACM-Reference-Format}
\bibliography{references}

\end{document}
