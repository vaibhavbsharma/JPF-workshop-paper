\documentclass[sigconf]{acmart}

\usepackage{booktabs} % For formal tables


% Copyright
%\setcopyright{none}
%\setcopyright{acmcopyright}
%\setcopyright{acmlicensed}
\setcopyright{rightsretained}
%\setcopyright{usgov}
%\setcopyright{usgovmixed}
%\setcopyright{cagov}
%\setcopyright{cagovmixed}


% DOI
% \acmDOI{10.475/123_4}

% ISBN
% \acmISBN{123-4567-24-567/08/06}

%Conference
% \acmConference[WOODSTOCK'97]{ACM Woodstock conference}{July 1997}{El
%   Paso, Texas USA} 
% \acmYear{1997}
% \copyrightyear{2016}
% 
% \acmPrice{15.00}


\begin{document}
\title{Veritesting Challenges in Symbolic Execution of Java Bytecode}
%\subtitle{Extended Abstract}
%\subtitlenote{The full version of the author's guide is available as
%  \texttt{acmart.pdf} document}


\author{Vaibhav Sharma}
%\authornote{Dr.~Trovato insisted his name be first.}
%\orcid{1234-5678-9012}
\affiliation{%
  \institution{University of Minnesota}
  %\streetaddress{P.O. Box 1212}
  \city{Minneapolis} 
  \state{MN} 
  \postcode{55455}
}
\email{vaibhav@umn.edu}

\author{Micheal Whalen}
%\authornote{The secretary disavows any knowledge of this author's actions.}
\affiliation{%
  \institution{University of Minnesota}
  %\streetaddress{P.O. Box 1212}
  \city{Minneapolis} 
  \state{MN} 
  \postcode{55455}
}
\email{whalen@umn.edu}

\author{Stephen McCamant}
%\authornote{The secretary disavows any knowledge of this author's actions.}
\affiliation{%
  \institution{University of Minnesota}
  %\streetaddress{P.O. Box 1212}
  \city{Minneapolis} 
  \state{MN} 
  \postcode{55455}
}
\email{mccamant@cs.umn.edu}

\author{Willem Visser}
\affiliation{%
  \institution{University of Stellenbosch}
  \city{Stellenbosch}
  \country{South Africa}
}
\email{wvisser@cs.sun.ac.za}
% The default list of authors is too long for headers}
\renewcommand{\shortauthors}{B. Trovato et al.}


\begin{abstract}
Scaling symbolic execution to industrial-sized code is an important open research problem. 
%
Veritesting is a promising technique that overcomes the scalability problem by combining the advantages of static symbolic execution with those of dynamic symbolic execution.
%
However, veritesting has been applied only to binary-level symbolic execution.
%
Integrating veritesting with symbolic execution of Java bytecode presents unique challenges.
%
Veritesting with Java bytecode requires incorporating non-local control jumps caused by runtime polymorphism, exceptions, native calls, and dynamic class loading.
%
Instead of incorporation, veritesting may be performed using such non-local control jumps as exit points
%
But, in practice, we find such regions of code to be small, thereby motivating the need to perform static symbolic execution of such non-local control jumps as often as possible.
%
We confirm this motivation by running a Soot-based static analysis on six large open-source projects used in the Defects4J collection.
%
We find that while veritesting can be applied in thousands of regions, allowing static symbolic execution of non-local control jumps has the potential of providing exponential speedup to dynamic symbolic execution.
%
\end{abstract}

%
% The code below should be generated by the tool at
% http://dl.acm.org/ccs.cfm
% Please copy and paste the code instead of the example below. 
%
% \begin{CCSXML}
% <ccs2012>
%  <concept>
%   <concept_id>10010520.10010553.10010562</concept_id>
%   <concept_desc>Computer systems organization~Embedded systems</concept_desc>
%   <concept_significance>500</concept_significance>
%  </concept>
%  <concept>
%   <concept_id>10010520.10010575.10010755</concept_id>
%   <concept_desc>Computer systems organization~Redundancy</concept_desc>
%   <concept_significance>300</concept_significance>
%  </concept>
%  <concept>
%   <concept_id>10010520.10010553.10010554</concept_id>
%   <concept_desc>Computer systems organization~Robotics</concept_desc>
%   <concept_significance>100</concept_significance>
%  </concept>
%  <concept>
%   <concept_id>10003033.10003083.10003095</concept_id>
%   <concept_desc>Networks~Network reliability</concept_desc>
%   <concept_significance>100</concept_significance>
%  </concept>
% </ccs2012>  
% \end{CCSXML}
% 
% \ccsdesc[500]{Computer systems organization~Embedded systems}
% \ccsdesc[300]{Computer systems organization~Redundancy}
% \ccsdesc{Computer systems organization~Robotics}
% \ccsdesc[100]{Networks~Network reliability}


\keywords{ACM proceedings, \LaTeX, text tagging}


\maketitle

\section{Introduction}
Introduce veritesting, present our simple example and its performance improvement, segue into how veritesting at Java bytecode level is different from binary level
\input{veritesting}
\section{Engineering Challenges}
Talk about the engineering challenges we face when implementing veritesting with Symbolic PathFinder

Engineering issue \#1: Sharing implementation needs to be fixed. Show this using the TestSharing example

Engineering issue \#2: Need to have complex expressions, talk about how Comparators cannot be used anywhere below the top-level operator

Engineering issue \#3: Nice to introduce intermediate variables
\section{Related Work}
Talk about veritesting in Mayhem, Angr.
Talk about other symbolic execution performance improvements.
%mentioned in Christopher Kruegel's ISSTA keynote talk as Static Analysis support
LESE - loop extended sym. exec. 
  - intelligent loop unrolling
Code summarization (Dodo)
  - automatically (and statically) summarize effect of loops / functions
VSA - value set analysis
  - resolve ranges (and conditionals) without solving constraints

Talk about TamiFlex, and how using the same technique as TamiFlex is one way to solve veritesting challenges in Java bytecode.
\section{Conclusion}
%Talk about veritesting is important, but implementing it at the Java bytecode requires solving different research and engineering challenges.
Veritesting provides a way to address the scalability challenges faced by dynamic symbolic execution.
%
Nevertheless, integrating veritesting with symbolic execution at the Java bytecode level presents unique challenges.
%
Our Soot-based analysis of six large open-source Java projects confirms that, while a number of opportunities to apply veritesting arise organically in Java bytecode, the benefit of veritesting can be amplified by adding ``just in time'' support of virtual function invocations and techniques for reducing the number of exit points introduced by exceptions and early return statements.  In addition, careful integration with Symbolic PathFinder is necessary to achieve good performance.
%
%Implementing veritesting with the Symbolic PathFinder motivates robust implementation of expression reuse and introduction of complex expressions.
%
%Using veritesting to its full extent during symbolic execution of Java bytecode requires us to address such research and engineering challenges.
%
%But, veritesting has clear potential to provide significant speed-up to symbolic execution tools such as Symbolic PathFinder.


\bibliographystyle{ACM-Reference-Format}
\bibliography{references} 

\end{document}
