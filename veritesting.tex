\section{Veritesting with Java bytecode}

While the performance improvement demonstrated on the code in Listing 1 is impressive, it is perhaps not representative of most Java code.  Java conventions encourage the use of indirection when accessing class fields using non-static {\em get} and {\em set} methods, as well as liberal use of exceptions.  Unlike C compilers, which assume a ``closed world'' and often inline simple functions into the body of calling methods to improve performance, the Java compiler must assume an ``open world'' in which a class may be used in a new context, so inlining of non-final methods is unsafe.  

Previous approaches to veritesting exit static code regions when indirect calls to functions or non-local jumps are made.  In this section, we explore how the structure of Java programs might reduce the performance of a n\"aive veritesting approach.

%Talk about how veritesting is different when done on Java bytecode
%
%The transition points at the Java bytecode level are return, exceptions (done using athrow), virtual function invocations (invokevirtual, invokeinterface), native calls (also done using invokevirtual), reflection and dynamic class loading (also done using invokevirtual).
%
%Present results from Soot-based static analysis
%
%Talk about how veritesting gets harder to do when considering multi-threaded Java bytecode.
\subsection{Exit points}
%While the performance improvement shown in Figure~\ref{fig:v_ex_plot} with statically unrolling the three-way branches is clear, we need to automate this process.
%
Integrating veritesting with SPF requires that we can represent a region of a java program as a disjunctive formula with multiple exit points.  Each exit point describes a possible distinct continuation of the current path after the static code region completes execution.  Avgerinos et al.~\cite{veritesting} defined exit points as unresolved jumps, function boundaries, and system calls.
%
These exit points are nodes in a Control Flow Graph~(CFG) which represent non-local control flow, and therefore, need to be explored using plain dynamic symbolic execution.
%
In the context of Java bytecode, we find such non-local control flow in five ways listed as follows: (1) return statements, (2) exceptions, (3) virtual function invocations~(\textit{invokevirtual}, \textit{invokeinterface}), (4) reflection, (5) native calls.
%
Return statements form the function boundary, and are obvious candidates to be considered as exit points.
%
Exceptions are used to catch unexpected behavior, and often help design of cleaner code, and allow developers to capture useful information when such errors occur.
%
Virtual function invocations occur due to runtime polymorphism supported by Java.
%
The runtime environment binds the method call to its body by using the type of the object making the call.
%
Using reflection requires loading a class at runtime, identifying a method within the class, and calling it.
%
It is primarily used for extensibility purposes, achieving separate compilation, and generating a class at runtime.
%
Exceptions, virtual function invocations, and reflection create non-local control flow.
%
An approach such as Avgerinos et al.~\cite{veritesting} would bound static code regions at each of these boundaries.


%Such regions must correctly preserve the semantics of symbolic execution for all possible Java constructs.
%
The primary benefit of implementing veritesting comes from its conversion of branches into disjunctions.
%
But this benefit exists only when the number of different exit points from the disjunctive formula is less than the number of execution paths through the region in the first place.
%
For example, all execution paths in the first three-way branch in Listing~\ref{lst:v_ex} joined together on line 14, causing the thre-way branch to have a single exit point.
%
Therefore, it is crucial for us to study the number of exit points for each of our statically-analyzed regions vs. the reduction in the number of branches within the region.
%
%
%Native calls are used to execute code outside of the Java Runtime Environment.
%
%Incorporating these into our static analysis would require handling virtual function invocations as well as static analysis of binary code.

These five types of exit points create the kind of non-local control flow which formed the frontier of the visible CFG created as a result of \textit{CFGReduce} step by Avgerinos et al.
%
However, many of these exit points are used pervasively by Java developers.
%
For example, the Visitor design pattern is used extensively by the ASM framework~\cite{asm}, Soot~\cite{soot} and makes use of Java\rq s dynamic dispatch mechanism.
%
Running into exit points too often causes our statically-analyzed regions to be small and our performance gain from having fewer branches to be lost.

We investigated the occurrence of these exit points further by creating a Soot-based static analysis of six large open-source projects written in Java.
%
Software faults from these six projects are maintained in the Defects4J~\cite{defects4j} repository.
%
We used Soot to create a CFG for every method body in every class file in these six projects.
%
For each CFG, we used nodes corresponding to \textit{if} bytecode instructions as a starting point of our analysis.
%
We measured the number of instructions encountered when traversing down each side of the branch until we get to an exit point.
%
While continuing to monitor for exit points, we continued our traversal of the CFG until we got to the immediate post dominator~\cite{dragon-book} of our starting point.
%
If there were no exit points encountered on any side of the branch, we considered this region as a pure veritesting region and calculated its size in bytecode instructions.
%
Finally, we allowed upto five nested branches and calculated the number of bytecode instructions from the earliest, as well as, the latest starting point in our CFG traversal to an exit point.
%
We report our results in Tables~\ref{t:r_s} and~\ref{t:r_c}.

\begin{table}[]
\centering
\caption{Soot-based analysis for number of bytecode instructions between starting and exit points}
\label{t:r_s}
\begin{tabular}{|l|l|l|l|l|l|}
\hline
        & \#classes & \begin{tabular}[c]{@{}l@{}}if-\\ return\end{tabular} & \begin{tabular}[c]{@{}l@{}}if-\\ invokevirtual\end{tabular} & \begin{tabular}[c]{@{}l@{}}if-\\ throw\end{tabular} & \begin{tabular}[c]{@{}l@{}}region \\ size\end{tabular} \\ \hline
chart   & 679       & 8.44                                                 & 27.47                                                       & 4.33                                                & 13.59                                                  \\ \hline
closure & 1339      & 7.35                                                 & 22.1                                                        & 9.5                                                 & 11.66                                                  \\ \hline
lang    & 170       & 6.70                                                 & 11.64                                                       & 7.09                                                & 9.60                                                   \\ \hline
math    & 1104      & 18.27                                                & 56.61                                                       & 9.56                                                & 27.06                                                  \\ \hline
mockito & 382       & 6.02                                                 & 12.51                                                       & 8.05                                                & 13.57                                                  \\ \hline
time    & 209       & 7.79                                                 & 13.10                                                       & 7.08                                                & 8.10                                                   \\ \hline
\end{tabular}
\end{table}

\begin{table}[]
\centering
\caption{Number of occurences in Soot-based static analysis}
\label{t:r_c}
\begin{tabular}{|l|l|l|l|l|}
\hline
        & \begin{tabular}[c]{@{}l@{}}if-\\ return\end{tabular} & \begin{tabular}[c]{@{}l@{}}if-\\ invokevirtual\end{tabular} & \begin{tabular}[c]{@{}l@{}}if-\\ throw\end{tabular} & \begin{tabular}[c]{@{}l@{}}region\\ count\end{tabular} \\ \hline
chart   & 1712                                                 & 7760                                                        & 521                                                 & 6627                                                   \\ \hline
closure & 3853                                                 & 7466                                                        & 138                                                 & 9258                                                   \\ \hline
lang    & 3602                                                 & 1589                                                        & 539                                                 & 2065                                                   \\ \hline
math    & 2219                                                 & 5582                                                        & 662                                                 & 15375                                                  \\ \hline
mockito & 372                                                  & 572                                                         & 15                                                  & 574                                                    \\ \hline
time    & 1202                                                 & 984                                                         & 204                                                 & 1421                                                   \\ \hline
\end{tabular}
\end{table}

Table~\ref{t:r_s} shows the average size and Table~\ref{t:r_c} shows the number of times each such count was reported.
\mike{We need to explain these tables better: can we take a small code fragment (say, that of Listing 1) and explain it according to these numbers, or create a very tiny code fragment that demonstrates them?}
%
The \textit{if-return}, \textit{if-invokevirtual}, \textit{if-throw} columns in Table~\ref{t:r_s} report the average number of instructions observed between any \textit{if} opcode-containing bytecode instruction and a \textit{return}, \textit{invokevirtual} or \textit{invokeinterface}, \textit{throw} opcode-containing bytecode instruction.
%
These same columns in Table~\ref{t:r_c} report the number of times we observed an occurence of one of \textit{return}, \textit{invokevirtual}, \textit{invokeinterface}, \textit{throw} opcode-containing bytecode instructions before reaching the immediate post-dominator of the starting \textit{if} node on any side of the branch.
%
We can interpret Table~\ref{t:r_s} better by noting that we covered 35 bytecode instructions in the veritesting region shown in Listing~\ref{lst:v_ex}.
%
Tables~\ref{t:r_s} and~\ref{t:r_c} show that while we discover thousands of regions which do not contain any exit points, these regions are small.
%
Having veritesting for these regions alone would provide a significant performance boost to SPF.
%
However, as shown by Table~\ref{t:r_c}, \textit{invokevirtual} or \textit{invokeinterface} instructions are encountered far more often than \textit{return} or \textit{throw} instructions.
%
This can be explained by the pervasive use of runtime polymorphism and exception handling by Java developers.
%
Instead of using \textit{invokevirtual} and \textit{invokeinterface} instructions as exit points, if we can continue our static symbolic execution beyond them, we would almost double the number of veritesting regions.
%
Tables~\ref{t:r_s} and~\ref{t:r_c} also show that early \textit{return} instructions are another often used construct in Java.
%
\subsection{Veritesting with Multi-threading}
Veritesting requires SPF to be able to perform static symbolic execution on a region of code and incorporate it as a disjunctive predicate into its path expression along with the corresponding updates to its symbolic store.
%
If the region being statically analyzed can be executed in a multi-threaded context, however, then it is necessary to consider all possible points of interference in the region.
%
This consideration requires changes to the updates made to SPF\rq s path expression and its symbolic store.
%
One way to handle this is to turn every point of interference as an exit point, but determining the possible points of interference statically is itself a difficult problem.  Likely it will require some level of dynamic analysis to determine the points of interference at the time of creation of a veritesting region.
%
%But, veritesting would be beneficial if its static analysis also includes computation of points at which interference is infeasible.
%
This makes creating an efficient veritesting approach challenging since the cost of this computation must be less than the cost of doing plain dynamic symbolic execution.
















% Data from perl script-based static analysis
% this script just counted the difference between instructions
%
% \begin{table}[]
% \centering
% \caption{My caption}
% \label{my-label}
% \begin{tabular}{|l|l|l|l|}
% \hline
%         & if-ret & if-IV & if-throw \\ \hline
% Chart   & 8.08   & 5.79  & 6.10     \\ \hline
% Closure & 7.40   & 4.37  & 12.6     \\ \hline
% Lang    & 6.3    & 5.23  & 8.26     \\ \hline
% Math    & 12.9   & 6.9   & 9.09     \\ \hline
% Mockito & 8.5    & 4.38  & 11.39    \\ \hline
% Time    & 9.0    & 5.56  & 9.3      \\ \hline
% \end{tabular}
% \end{table}
